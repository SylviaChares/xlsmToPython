\documentclass[11pt,a4paper]{article}

% Pakete
\usepackage[utf8]{inputenc}
\usepackage[ngerman]{babel}
\usepackage[T1]{fontenc}
\usepackage{amsmath}
\usepackage{amssymb}
\usepackage{geometry}
\usepackage{fancyhdr}
\usepackage{graphicx}
\usepackage{booktabs}
\usepackage{enumitem}
\usepackage{xcolor}
\usepackage{listings}
\usepackage{hyperref}

% Seitenlayout
\geometry{left=2.5cm, right=2.5cm, top=2.5cm, bottom=2.5cm}
\pagestyle{fancy}
\fancyhf{}
\fancyhead[L]{Tarifrechner KLV -- Strukturanalyse}
\fancyhead[R]{\thepage}
\renewcommand{\headrulewidth}{0.4pt}

% Farben
\definecolor{codegreen}{rgb}{0,0.6,0}
\definecolor{codegray}{rgb}{0.5,0.5,0.5}
\definecolor{codepurple}{rgb}{0.58,0,0.82}
\definecolor{backcolour}{rgb}{0.95,0.95,0.92}

% Code-Listings
\lstdefinestyle{vbastyle}{
    backgroundcolor=\color{backcolour},
    commentstyle=\color{codegreen},
    keywordstyle=\color{blue},
    numberstyle=\tiny\color{codegray},
    stringstyle=\color{codepurple},
    basicstyle=\ttfamily\small,
    breakatwhitespace=false,
    breaklines=true,
    captionpos=b,
    keepspaces=true,
    numbers=left,
    numbersep=5pt,
    showspaces=false,
    showstringspaces=false,
    showtabs=false,
    tabsize=2
}

\lstset{style=vbastyle}

% Titel
\title{\textbf{Strukturanalyse} \\[0.5em] 
       \Large Tarifrechner für Kapitallebensversicherung \\[0.3em]
       \large Tarifrechner\_KLV.xlsm}
\author{Aktuarielle Analyse für Python-Migration}
\date{\today}

\begin{document}

\maketitle
\thispagestyle{empty}

\vfill

\begin{abstract}
Diese Dokumentation analysiert die Struktur und Berechnungslogik des aktuariellen Tarifrechners \texttt{Tarifrechner\_KLV.xlsm} für Kapitallebensversicherungen. Die Analyse dient als Grundlage für die Migration des Excel-basierten Systems nach Python. Das Dokument beschreibt die Datenstruktur, VBA-Module, versicherungsmathematischen Formeln sowie kritische Punkte für die Implementierung.
\end{abstract}

\vfill

\clearpage
\tableofcontents
\clearpage

%===============================================================================
\section{Übersicht}
%===============================================================================

\subsection{Dateiinformationen}

\begin{itemize}[itemsep=0.5em]
    \item \textbf{Datei:} \texttt{Tarifrechner\_KLV.xlsm}
    \item \textbf{Typ:} Excel Macro-Enabled Workbook
    \item \textbf{Zweck:} Aktuarieller Tarifrechner für Kapitallebensversicherung (KLV)
    \item \textbf{Berechnungsbasis:} DAV-Sterbetafeln (Deutsche Aktuarvereinigung)
    \item \textbf{Tafeln:} DAV1994\_T, DAV2008\_T (jeweils männlich/weiblich)
\end{itemize}

\subsection{Produktbeschreibung}

Der Tarifrechner berechnet Beiträge und Deckungsrückstellungen für eine Kapitallebensversicherung mit folgenden Charakteristika:

\begin{itemize}
    \item Kombinierte Todes- und Erlebensfallleistung
    \item Temporäre Beitragszahlung (Dauer $t \leq n$)
    \item Versicherungsdauer $n$ Jahre
    \item Unterjährige Beitragszahlung möglich (jährlich, halbjährlich, vierteljährlich, monatlich)
    \item Kostenbelastung: Abschluss-, Inkasso- und Verwaltungskosten
\end{itemize}

%===============================================================================
\section{Tabellenblatt-Struktur}
%===============================================================================

\subsection{Tabellenblatt \enquote{Kalkulation}}

\subsubsection{Übersicht}

\begin{itemize}
    \item \textbf{Dimensionen:} A1:L66
    \item \textbf{Maximale Zeile:} 66
    \item \textbf{Maximale Spalte:} 12 (A--L)
    \item \textbf{Anzahl Formeln:} 309
    \item \textbf{Typ:} Berechnungsblatt mit Eingaben und Ergebnissen
\end{itemize}

\subsubsection{Eingabebereich: Vertragsdaten}

Spalten A--B, Zeilen 4--9:

\begin{table}[h]
\centering
\begin{tabular}{lll}
\toprule
\textbf{Parameter} & \textbf{Symbol} & \textbf{Beispielwert} \\
\midrule
Eintrittsalter & $x$ & 40 Jahre \\
Geschlecht & Sex & M (männlich) \\
Versicherungsdauer & $n$ & 30 Jahre \\
Beitragszahlungsdauer & $t$ & 20 Jahre \\
Versicherungssumme & VS & 100.000 EUR \\
Zahlungsweise & zw & 12 (monatlich) \\
\bottomrule
\end{tabular}
\caption{Vertragsdaten -- Eingabeparameter}
\end{table}

\textbf{Zulässige Werte für Zahlungsweise (zw):}
\begin{itemize}
    \item $\text{zw} = 1$: jährliche Zahlungsweise
    \item $\text{zw} = 2$: halbjährliche Zahlungsweise  
    \item $\text{zw} = 4$: vierteljährliche Zahlungsweise
    \item $\text{zw} = 12$: monatliche Zahlungsweise
\end{itemize}

\subsubsection{Eingabebereich: Tarifdaten}

Spalten D--E, Zeilen 4--12:

\begin{table}[h]
\centering
\begin{tabular}{lll}
\toprule
\textbf{Parameter} & \textbf{Symbol} & \textbf{Beispielwert} \\
\midrule
Rechnungszins & $i$ & 1,75\% \\
Sterbetafel & Tafel & DAV1994\_T \\
Abschlusskostensatz & $\alpha$ & 2,50\% \\
Inkassokostensatz & $\beta_1$ & 2,50\% \\
Verwaltungskostensatz Tod & $\gamma_1$ & 0,08\% \\
Verwaltungskostensatz Erlebensfall & $\gamma_2$ & 0,125\% \\
Verwaltungskostensatz DR & $\gamma_3$ & 0,25\% \\
Stückkosten & $k$ & 24,00 EUR \\
Ratenzuschlag & ratzu & 5\% \\
\bottomrule
\end{tabular}
\caption{Tarifdaten -- Kostensätze und Parameter}
\end{table}

\textbf{Hinweis zu Stückkosten:} Der Parameter $k$ (Zelle E11) bezeichnet die Stückkosten in EUR, nicht die Anzahl Zahlungen pro Jahr.

\textbf{Ratenzuschlag:} Wird per Formel aus Zahlungsweise berechnet:
\begin{equation}
\text{ratzu} = \begin{cases}
2\% & \text{falls } zw = 2 \\
3\% & \text{falls } zw = 4 \\
5\% & \text{falls } zw = 12 \\
0\% & \text{sonst}
\end{cases}
\end{equation}

\subsubsection{Eingabebereich: Grenzen für flexible Phase}

Spalten G--H, Zeilen 4--5:

\begin{table}[h]
\centering
\begin{tabular}{lll}
\toprule
\textbf{Parameter} & \textbf{Symbol} & \textbf{Wert} \\
\midrule
Mindestalter für flexible Phase & MinAlterFlex & 60 Jahre \\
Mindestrestlaufzeit für flexible Phase & MinRLZFlex & 5 Jahre \\
\bottomrule
\end{tabular}
\caption{Grenzen für flexible Phase}
\end{table}

\subsubsection{Beitragsberechnungen}

Spalten J--K, Zeilen 5--9:

\begin{table}[h]
\centering
\begin{tabular}{lll}
\toprule
\textbf{Größe} & \textbf{Formel} & \textbf{Zelle} \\
\midrule
Bruttobeitrag Tod (je 1 EUR VS) & $B_{x:t}$ & K5 (Array-Formel) \\
Jahresbeitrag & BJB & K6: $= \text{VS} \cdot B_{x:t}$ \\
Zahlbeitrag & BZB & K7: $= \frac{1+\text{ratzu}}{zw} \cdot (\text{BJB} + k)$ \\
Nettoprämiensatz Tod & $P_{x:t}$ & K9 (Array-Formel) \\
\bottomrule
\end{tabular}
\caption{Beitragsberechnungen}
\end{table}

\textbf{Array-Formel für Bruttobeitrag Tod} (Zelle K5):

\begin{multline}
B_{x:t} = \frac{{}_{n}A_x + {}_{n}E_x + \gamma_1 \cdot \ddot{a}_{x:t}^{(1)} + \gamma_2 \cdot \left(\ddot{a}_{x:n}^{(1)} - \ddot{a}_{x:t}^{(1)}\right)}{(1-\beta_1) \cdot \ddot{a}_{x:t}^{(1)} - \alpha \cdot t}
\end{multline}

wobei:
\begin{itemize}
    \item ${}_{n}A_x$ = Barwert temporäre Todesfallversicherung
    \item ${}_{n}E_x$ = Barwert reine Erlebensfallversicherung
    \item $\ddot{a}_{x:t}^{(1)}$ = Barwert temporäre vorschüssige Leibrente (jährlich)
    \item $\ddot{a}_{x:n}^{(1)}$ = Barwert temporäre vorschüssige Leibrente (jährlich, $n$ Jahre)
\end{itemize}

\textbf{Array-Formel für Nettoprämiensatz Tod} (Zelle K9):

\begin{equation}
P_{x:t} = \frac{{}_{n}A_x + {}_{n}E_x + t \cdot \alpha \cdot B_{x:t}}{\ddot{a}_{x:t}^{(1)}}
\end{equation}

\subsubsection{Verlaufswerte}

Zeilen 14--66: Berechnung versicherungsmathematischer Größen für jedes Vertragsjahr $k$.

\textbf{Spaltenüberschriften} (Zeile 15):

\begin{table}[h]
\centering
\small
\begin{tabular}{lll}
\toprule
\textbf{Spalte} & \textbf{Bezeichnung} & \textbf{Beschreibung} \\
\midrule
A & $k$ & Vertragsjahr \\
B & ${}_{n-k}A_{x+k}$ & Todesfallbarwert ab Jahr $k$ \\
C & $\ddot{a}_{x+k:n-k}^{(1)}$ & Rentenbarwert gesamt ab Jahr $k$ \\
D & $\ddot{a}_{x+k:t-k}^{(1)}$ & Rentenbarwert Beitragszahlung ab Jahr $k$ \\
E & ${}_kV_x^{\text{bpfl}}$ & Deckungsrückstellung beitragspflichtig (je 1 EUR VS) \\
F & ${}_kDR_x^{\text{bpfl}}$ & Deckungsrückstellung beitragspflichtig (EUR) \\
G & ${}_kV_x^{\text{bfr}}$ & Deckungsrückstellung beitragsfrei (je 1 EUR VS) \\
H & ${}_kV_x^{\text{MRV}}$ & Deckungsrückstellung Mindestrückkaufswert (je 1 EUR VS) \\
I & -- & Flexible Phase (Kennzeichen) \\
J & StoAb & Stornoabzug \\
K & RKW & Rückkaufswert \\
L & $\text{VS}_{\text{bfr}}$ & Versicherungssumme beitragsfrei \\
\bottomrule
\end{tabular}
\caption{Verlaufswerte -- Spaltenüberschriften}
\end{table}

\textbf{Zentrale Array-Formeln für Verlaufswerte:}

\begin{align}
{}_{n-k}A_{x+k} &= 
\begin{cases}
\frac{M_{x+k} - M_{x+n}}{D_{x+k}} + \frac{D_{x+n}}{D_{x+k}} & \text{falls } k \leq n \\
0 & \text{sonst}
\end{cases}
\label{eq:axn_verlauf} \\
%
\ddot{a}_{x+k:n-k}^{(1)} &= \frac{N_{x+k} - N_{x+n}}{D_{x+k}} - \beta(1, i) \cdot \left(1 - \frac{D_{x+n}}{D_{x+k}}\right) 
\label{eq:axn_k_verlauf} \\
%
\ddot{a}_{x+k:t-k}^{(1)} &= \frac{N_{x+k} - N_{x+t}}{D_{x+k}} - \beta(1, i) \cdot \left(1 - \frac{D_{x+t}}{D_{x+k}}\right)
\label{eq:axt_k_verlauf}
\end{align}

\textbf{Deckungsrückstellung beitragspflichtig:}

\begin{multline}
{}_kV_x^{\text{bpfl}} = {}_{n-k}A_{x+k} - P_{x:t} \cdot \ddot{a}_{x+k:t-k}^{(1)} \\
+ \gamma_2 \cdot \left(\ddot{a}_{x+k:n-k}^{(1)} - \frac{\ddot{a}_{x:n}^{(1)}}{\ddot{a}_{x:t}^{(1)}} \cdot \ddot{a}_{x+k:t-k}^{(1)}\right)
\label{eq:kvx_bpfl}
\end{multline}

\textbf{Deckungsrückstellung beitragsfrei:}

\begin{equation}
{}_kV_x^{\text{bfr}} = {}_{n-k}A_{x+k} + \gamma_3 \cdot \ddot{a}_{x+k:n-k}^{(1)}
\label{eq:kvx_bfr}
\end{equation}

%===============================================================================
\subsection{Tabellenblatt \enquote{Tafeln}}
%===============================================================================

\subsubsection{Übersicht}

\begin{itemize}
    \item \textbf{Dimensionen:} A3:E127
    \item \textbf{Maximale Zeile:} 127
    \item \textbf{Maximale Spalte:} 5 (A--E)
    \item \textbf{Anzahl Formeln:} 0
    \item \textbf{Typ:} Reines Datenblatt (Sterbewahrscheinlichkeiten)
\end{itemize}

\subsubsection{Struktur}

\textbf{Header} (Zeile 3):

\begin{table}[h]
\centering
\begin{tabular}{ll}
\toprule
\textbf{Spalte} & \textbf{Bezeichnung} \\
\midrule
A & $x/y$ (Alter) \\
B & DAV1994\_T\_M (DAV 1994, männlich) \\
C & DAV1994\_T\_F (DAV 1994, weiblich) \\
D & DAV2008\_T\_M (DAV 2008, männlich) \\
E & DAV2008\_T\_F (DAV 2008, weiblich) \\
\bottomrule
\end{tabular}
\caption{Tafeln -- Spaltenstruktur}
\end{table}

\textbf{Daten} (Zeilen 4--127):
\begin{itemize}
    \item Alter: $x \in \{0, 1, 2, \ldots, 123\}$
    \item Sterbewahrscheinlichkeiten $q_x$ für jede Tafel
    \item Letztes Alter ($x=123$): $q_{123} = 1{,}0$ (sicherer Tod)
\end{itemize}

\textbf{Beispiel-Daten:}

\begin{table}[h]
\centering
\begin{tabular}{lrrrr}
\toprule
$x$ & \textbf{DAV1994\_T\_M} & \textbf{DAV1994\_T\_F} & \textbf{DAV2008\_T\_M} & \textbf{DAV2008\_T\_F} \\
\midrule
0   & 0,011687 & 0,009003 & 0,006113 & 0,005088 \\
40  & 0,002378 & 0,001233 & 0,001618 & 0,000680 \\
60  & 0,007891 & 0,003932 & 0,005913 & 0,002861 \\
123 & 1,000000 & 1,000000 & 1,000000 & 1,000000 \\
\bottomrule
\end{tabular}
\caption{Beispiel-Sterbewahrscheinlichkeiten}
\end{table}

%===============================================================================
\section{VBA-Modul-Struktur}
%===============================================================================

\subsection{Modul \enquote{mConstants}}

\subsubsection{Zweck}
Definiert globale Konstanten für Rundung und Altersgrenzen.

\subsubsection{Konstanten}

\begin{lstlisting}[language=vba, caption={VBA-Code: Modul mConstants}]
Public Const rund_lx As Integer = 16    ' Rundung fuer Lebende
Public Const rund_tx As Integer = 16    ' Rundung fuer Tote
Public Const rund_Dx As Integer = 16    ' Rundung fuer Dx-Werte
Public Const rund_Cx As Integer = 16    ' Rundung fuer Cx-Werte
Public Const rund_Nx As Integer = 16    ' Rundung fuer Nx-Werte
Public Const rund_Mx As Integer = 16    ' Rundung fuer Mx-Werte
Public Const rund_Rx As Integer = 16    ' Rundung fuer Rx-Werte
Public Const max_Alter As Integer = 123 ' Maximales Alter
\end{lstlisting}

\textbf{Erläuterung:}
\begin{itemize}
    \item Alle Rundungen auf 16 Dezimalstellen
    \item Höchstalter 123 Jahre (entspricht Tafel-Maximum)
    \item Verwendung von \texttt{WorksheetFunction.Round} mit Banker's Rounding
\end{itemize}

%===============================================================================
\subsection{Modul \enquote{mGWerte}}
%===============================================================================

\subsubsection{Zweck}
Berechnung aktuarieller Grundwerte (Commutation Functions).

\subsubsection{Sterbewahrscheinlichkeiten}

\textbf{Funktion:} \texttt{Act\_qx(Alter, Sex, Tafel, ...)}

\begin{itemize}
    \item Liest Sterbewahrscheinlichkeit aus Tabellenblatt \enquote{Tafeln}
    \item Unterstützte Tafeln: DAV1994\_T, DAV2008\_T
    \item Geschlecht: M (männlich), F (weiblich)
    \item Gibt $q_x$-Wert zurück
\end{itemize}

\subsubsection{Anzahl Lebende ($l_x$-Werte)}

\textbf{Rekursive Berechnung:}

\begin{align}
l_0 &= 1\,000\,000 \\
l_{x+1} &= l_x \cdot (1 - q_x) \quad \text{für } x = 0, 1, \ldots, 122
\end{align}

Nach jeder Berechnung: Rundung auf \texttt{rund\_lx} = 16 Dezimalstellen.

\textbf{Funktionen:}
\begin{itemize}
    \item \texttt{v\_lx(Endalter, Sex, Tafel, ...)} -- Private Funktion, erzeugt Vektor
    \item \texttt{Act\_lx(Alter, Sex, Tafel, ...)} -- Public Funktion, gibt Einzelwert zurück
\end{itemize}

\subsubsection{Anzahl Tote ($t_x$-Werte)}

\textbf{Berechnung:}

\begin{equation}
t_x = l_x - l_{x+1}
\end{equation}

\textbf{Funktionen:}
\begin{itemize}
    \item \texttt{v\_tx(Endalter, Sex, Tafel, ...)} -- Private Funktion, erzeugt Vektor
    \item \texttt{Act\_tx(Alter, Sex, Tafel, ...)} -- Public Funktion, gibt Einzelwert zurück
\end{itemize}

\subsubsection{Kommutationswerte $D_x$}

\textbf{Berechnung:}

\begin{equation}
D_x = l_x \cdot v^x \quad \text{mit } v = \frac{1}{1+i}
\end{equation}

wobei $v$ der Diskontierungsfaktor und $i$ der Rechnungszins ist.

\textbf{Funktionen:}
\begin{itemize}
    \item \texttt{v\_Dx(Endalter, Sex, Tafel, Zins, ...)} -- Private Funktion, erzeugt Vektor
    \item \texttt{Act\_Dx(Alter, Sex, Tafel, Zins, ...)} -- Public Funktion mit Cache
\end{itemize}

\textbf{Cache-Mechanismus:}
\begin{itemize}
    \item Prüft Dictionary-Cache vor Berechnung
    \item Speichert Ergebnis im Cache
    \item Cache-Key: \enquote{Dx\_Alter\_Sex\_Tafel\_Zins\_...}
\end{itemize}

\subsubsection{Kommutationswerte $C_x$}

\textbf{Berechnung:}

\begin{equation}
C_x = t_x \cdot v^{x+1}
\end{equation}

\textbf{Funktionen:}
\begin{itemize}
    \item \texttt{v\_Cx(Endalter, Sex, Tafel, Zins, ...)} -- Private Funktion, erzeugt Vektor
    \item \texttt{Act\_Cx(Alter, Sex, Tafel, Zins, ...)} -- Public Funktion mit Cache
\end{itemize}

\subsubsection{Kommutationswerte $N_x$}

\textbf{Berechnung (rückwärts):}

\begin{align}
N_{\omega} &= D_{\omega} \quad \text{mit } \omega = 123 \\
N_x &= N_{x+1} + D_x \quad \text{für } x = \omega-1, \omega-2, \ldots, 0
\end{align}

Mathematisch äquivalent zu:
\begin{equation}
N_x = \sum_{j=x}^{\omega} D_j
\end{equation}

\textbf{Funktionen:}
\begin{itemize}
    \item \texttt{v\_Nx(Sex, Tafel, Zins, ...)} -- Private Funktion, erzeugt Vektor
    \item \texttt{Act\_Nx(Alter, Sex, Tafel, Zins, ...)} -- Public Funktion mit Cache
\end{itemize}

\subsubsection{Kommutationswerte $M_x$}

\textbf{Berechnung (analog zu $N_x$):}

\begin{align}
M_{\omega} &= C_{\omega} \\
M_x &= M_{x+1} + C_x
\end{align}

Mathematisch:
\begin{equation}
M_x = \sum_{j=x}^{\omega} C_j
\end{equation}

\textbf{Funktionen:}
\begin{itemize}
    \item \texttt{v\_Mx(Sex, Tafel, Zins, ...)} -- Private Funktion, erzeugt Vektor
    \item \texttt{Act\_Mx(Alter, Sex, Tafel, Zins, ...)} -- Public Funktion mit Cache
\end{itemize}

\subsubsection{Kommutationswerte $R_x$}

\textbf{Berechnung (analog zu $N_x$):}

\begin{align}
R_{\omega} &= M_{\omega} \\
R_x &= R_{x+1} + M_x
\end{align}

Mathematisch:
\begin{equation}
R_x = \sum_{j=x}^{\omega} M_j
\end{equation}

\textbf{Funktionen:}
\begin{itemize}
    \item \texttt{v\_Rx(Sex, Tafel, Zins, ...)} -- Private Funktion, erzeugt Vektor
    \item \texttt{Act\_Rx(Alter, Sex, Tafel, Zins, ...)} -- Public Funktion mit Cache
\end{itemize}

\subsubsection{Altersberechnung}

\textbf{Funktion:} \texttt{Act\_Altersberechnung(GebDat, BerDat, Methode)}

\begin{itemize}
    \item Methode \enquote{K}: Kalenderjahresmethode
    \begin{equation}
    \text{Alter} = \text{Jahr}_{\text{Ber}} - \text{Jahr}_{\text{Geb}}
    \end{equation}
    
    \item Methode \enquote{H}: Halbjahresmethode (Standard)
    \begin{equation}
    \text{Alter} = \left\lfloor \text{Jahr}_{\text{Ber}} - \text{Jahr}_{\text{Geb}} + \frac{1}{12} \cdot (\text{Monat}_{\text{Ber}} - \text{Monat}_{\text{Geb}} + 5) \right\rfloor
    \end{equation}
\end{itemize}

\subsubsection{Cache-Management}

\textbf{Initialisierung:}

\begin{lstlisting}[language=vba, caption={Cache-Initialisierung}]
Sub InitializeCache()
    Set cache = CreateObject("Scripting.Dictionary")
End Sub
\end{lstlisting}

\textbf{Cache-Key-Generierung:}

\begin{lstlisting}[language=vba, caption={Cache-Key-Funktion}]
Private Function CreateCacheKey(Art As String, Alter As Integer, _
                               Sex As String, Tafel As String, _
                               Zins As Double, GebJahr As Integer, _
                               Rentenbeginnalter As Integer, _
                               Schicht As Integer) As String
    CreateCacheKey = Art & "_" & Alter & "_" & Sex & "_" & _
                     Tafel & "_" & Zins & "_" & GebJahr & "_" & _
                     Rentenbeginnalter & "_" & Schicht
End Function
\end{lstlisting}

\textbf{Cache-Strategie:}
\begin{itemize}
    \item Vermeidet Mehrfachberechnung identischer Werte
    \item Wichtig bei Array-Formeln mit vielen Funktionsaufrufen
    \item Cache existiert nur während Excel-Sitzung
\end{itemize}

%===============================================================================
\subsection{Modul \enquote{mBarwerte}}
%===============================================================================

\subsubsection{Zweck}
Berechnung versicherungsmathematischer Barwerte (Leibrenten, Versicherungen).

\subsubsection{Lebenslange Leibrente}

\textbf{Funktion:} \texttt{Act\_ax\_k(Alter, Sex, Tafel, Zins, k, ...)}

Berechnet Barwert einer lebenslangen vorschüssigen Leibrente mit $k$ Zahlungen pro Jahr.

\textbf{Formel:}
\begin{equation}
\ddot{a}_x^{(k)} = \frac{N_x}{D_x} - \beta(k, i)
\end{equation}

wobei $\beta(k, i)$ das Abzugsglied für unterjährige Zahlungen ist.

\subsubsection{Temporäre Leibrente}

\textbf{Funktion:} \texttt{Act\_axn\_k(Alter, n, Sex, Tafel, Zins, k, ...)}

Berechnet Barwert einer temporären vorschüssigen Leibrente ($n$ Jahre, $k$ Zahlungen pro Jahr).

\textbf{Formel:}
\begin{equation}
\ddot{a}_{x:\overline{n}\,}^{(k)} = \frac{N_x - N_{x+n}}{D_x} - \beta(k, i) \cdot \left(1 - \frac{D_{x+n}}{D_x}\right)
\end{equation}

\subsubsection{Aufgeschobene Leibrente}

\textbf{Funktion:} \texttt{Act\_nax\_k(Alter, n, Sex, Tafel, Zins, k, ...)}

Berechnet Barwert einer aufgeschobenen vorschüssigen Leibrente (Beginn nach $n$ Jahren).

\textbf{Formel:}
\begin{equation}
{}_n\ddot{a}_x^{(k)} = \frac{D_{x+n}}{D_x} \cdot \ddot{a}_{x+n}^{(k)}
\end{equation}

\subsubsection{Todesfallbarwert (temporär)}

\textbf{Funktion:} \texttt{Act\_nGrAx(Alter, n, Sex, Tafel, Zins, ...)}

Berechnet Barwert einer temporären Todesfallversicherung (Leistung am Jahresende).

\textbf{Formel:}
\begin{equation}
{}_nA_x = \frac{M_x - M_{x+n}}{D_x}
\end{equation}

\subsubsection{Erlebensfallbarwert}

\textbf{Funktion:} \texttt{Act\_nGrEx(Alter, n, Sex, Tafel, Zins, ...)}

Berechnet Barwert einer reinen Erlebensfallversicherung (Leistung nach $n$ Jahren).

\textbf{Formel:}
\begin{equation}
{}_nE_x = \frac{D_{x+n}}{D_x}
\end{equation}

\subsubsection{Endliche Rente}

\textbf{Funktion:} \texttt{Act\_ag\_k(g, Zins, k)}

Berechnet Barwert einer endlichen vorschüssigen Rente ($g$ Zahlungen, kein Todesfallrisiko).

\textbf{Formel:}

\begin{equation}
\ddot{a}_{\overline{g}\,}^{(k)} = 
\begin{cases}
\frac{1 - v^g}{1 - v} - \beta(k, i) \cdot (1 - v^g) & \text{falls } i > 0 \\
g & \text{falls } i = 0
\end{cases}
\end{equation}

\subsubsection{Abzugsglied für unterjährige Zahlungen}

\textbf{Funktion:} \texttt{Act\_Abzugsglied(k, Zins)}

Korrektur für unterjährige Rentenzahlungen nach Woolhouse-Näherung (1. Ordnung).

\textbf{Formel:}

\begin{equation}
\beta(k, i) = \frac{1+i}{k} \sum_{\ell=0}^{k-1} \frac{\ell/k}{1 + (\ell/k) \cdot i}
\end{equation}

Für $k=1$ (jährliche Zahlungen): $\beta(1, i) = 0$.

\textbf{Vereinfachte Woolhouse-Formel:}

Für praktische Anwendungen oft genähert als:
\begin{equation}
\beta(k, i) \approx \frac{k-1}{2k}
\end{equation}

Diese Näherung ist für kleine Zinssätze ausreichend genau.

%===============================================================================
\section{Definierte Namen (Named Ranges)}
%===============================================================================

Die Excel-Arbeitsmappe verwendet 24 definierte Namen für zentrale Parameter und Bereiche.

\subsection{Eingabeparameter}

\begin{table}[h]
\centering
\begin{tabular}{lll}
\toprule
\textbf{Name} & \textbf{Zellbezug} & \textbf{Beschreibung} \\
\midrule
x & Kalkulation!\$B\$4 & Eintrittsalter \\
Sex & Kalkulation!\$B\$5 & Geschlecht (M/F) \\
n & Kalkulation!\$B\$6 & Versicherungsdauer \\
t & Kalkulation!\$B\$7 & Beitragszahlungsdauer \\
VS & Kalkulation!\$B\$8 & Versicherungssumme \\
zw & Kalkulation!\$B\$9 & Zahlungsweise \\
\bottomrule
\end{tabular}
\caption{Definierte Namen -- Eingabeparameter Vertragsdaten}
\end{table}

\subsection{Tarifdaten}

\begin{table}[h]
\centering
\begin{tabular}{lll}
\toprule
\textbf{Name} & \textbf{Zellbezug} & \textbf{Beschreibung} \\
\midrule
Zins & Kalkulation!\$E\$4 & Rechnungszins $i$ \\
Tafel & Kalkulation!\$E\$5 & Sterbetafel \\
alpha & Kalkulation!\$E\$6 & Abschlusskostensatz $\alpha$ \\
beta1 & Kalkulation!\$E\$7 & Inkassokostensatz $\beta_1$ \\
gamma1 & Kalkulation!\$E\$8 & Verwaltungskostensatz Tod $\gamma_1$ \\
gamma2 & Kalkulation!\$E\$9 & Verwaltungskostensatz Erlebensfall $\gamma_2$ \\
gamma3 & Kalkulation!\$E\$10 & Verwaltungskostensatz DR $\gamma_3$ \\
k & Kalkulation!\$E\$11 & Stückkosten \\
ratzu & Kalkulation!\$E\$12 & Ratenzuschlag \\
\bottomrule
\end{tabular}
\caption{Definierte Namen -- Tarifdaten}
\end{table}

\subsection{Grenzen und berechnete Werte}

\begin{table}[h]
\centering
\begin{tabular}{lll}
\toprule
\textbf{Name} & \textbf{Zellbezug} & \textbf{Beschreibung} \\
\midrule
MinAlterFlex & Kalkulation!\$H\$4 & Mindestalter flexible Phase \\
MinRLZFlex & Kalkulation!\$H\$5 & Mindestrestlaufzeit flexible Phase \\
B\_xt & Kalkulation!\$K\$5 & Bruttobeitrag Tod $B_{x:t}$ \\
BJB & Kalkulation!\$K\$6 & Jahresbeitrag \\
P\_xt & Kalkulation!\$K\$9 & Nettoprämiensatz Tod $P_{x:t}$ \\
\bottomrule
\end{tabular}
\caption{Definierte Namen -- Grenzen und Berechnungen}
\end{table}

\subsection{Tafel-Bereiche}

\begin{table}[h]
\centering
\begin{tabular}{lll}
\toprule
\textbf{Name} & \textbf{Zellbezug} & \textbf{Beschreibung} \\
\midrule
v\_Tafeln & Tafeln!\$B\$3:\$E\$3 & Tafelbezeichnungen (Header) \\
m\_Tafeln & Tafeln!\$B\$4:\$E\$127 & $q_x$-Werte (Matrix) \\
v\_x & Tafeln!\$A\$4:\$A\$127 & Alter $x = 0, \ldots, 123$ \\
\bottomrule
\end{tabular}
\caption{Definierte Namen -- Tafel-Bereiche}
\end{table}

%===============================================================================
\section{Berechnungslogik und Abhängigkeiten}
%===============================================================================

\subsection{Berechnungskette}

Die Berechnung erfolgt in folgender hierarchischer Reihenfolge:

\begin{enumerate}
    \item \textbf{Eingabe:} Vertragsdaten ($x$, Sex, $n$, $t$, VS, zw) und Tarifdaten ($i$, Tafel, Kostensätze)
    
    \item \textbf{VBA mGWerte:} Grundwerte
    \begin{itemize}
        \item $q_x$-Werte aus Sterbetafeln
        \item $l_x$ (Überlebende): $l_0 = 1\,000\,000$, $l_{x+1} = l_x \cdot (1-q_x)$
        \item Kommutationswerte: $D_x$, $C_x$, $N_x$, $M_x$, $R_x$
        \item Mit Cache-Mechanismus
    \end{itemize}
    
    \item \textbf{VBA mBarwerte:} Barwerte
    \begin{itemize}
        \item Leibrenten: $\ddot{a}_x^{(k)}$, $\ddot{a}_{x:\overline{n}\,}^{(k)}$
        \item Versicherungsbarwerte: ${}_nA_x$, ${}_nE_x$
    \end{itemize}
    
    \item \textbf{Excel-Formel K5:} Beitrag $B_{x:t}$
    \begin{itemize}
        \item Zähler: Todesfallbarwert $+$ Erlebensfallbarwert $+$ Kostenzuschläge
        \item Nenner: Beitragszahlungsbarwert abzüglich Abschlusskosten
    \end{itemize}
    
    \item \textbf{Excel-Formeln K6--K7:} BJB, BZB (mit Ratenzuschlag)
    
    \item \textbf{Excel-Formel K9:} Nettoprämiensatz $P_{x:t}$
    
    \item \textbf{Excel-Formeln Zeilen 16+:} Verlaufswerte pro Vertragsjahr
    \begin{itemize}
        \item Barwerte: ${}_{n-k}A_{x+k}$, $\ddot{a}_{x+k:n-k}^{(1)}$, $\ddot{a}_{x+k:t-k}^{(1)}$
        \item Deckungsrückstellungen: ${}_kV_x^{\text{bpfl}}$, ${}_kV_x^{\text{bfr}}$
        \item Rückkaufswerte: RKW
    \end{itemize}
\end{enumerate}

\subsection{Datenflussdiagramm}

\begin{figure}[h]
\centering
\begin{minipage}{0.9\textwidth}
\small
\begin{verbatim}
┌─────────────────────────────────────────────────────────┐
│  EINGABE: Vertragsdaten & Tarifdaten (Kalkulation)     │
└────────────────┬────────────────────────────────────────┘
                 │
                 v
┌─────────────────────────────────────────────────────────┐
│  STERBETAFELN: qx-Werte (Tafeln B4:E127)               │
└────────────────┬────────────────────────────────────────┘
                 │
                 v
┌─────────────────────────────────────────────────────────┐
│  VBA mGWerte: Berechnung Grundwerte                    │
│   - lx (Überlebende)                                   │
│   - Dx, Cx, Nx, Mx, Rx (Kommutationswerte)            │
│   - Mit Cache-Mechanismus                              │
└────────────────┬────────────────────────────────────────┘
                 │
                 v
┌─────────────────────────────────────────────────────────┐
│  VBA mBarwerte: Berechnung Barwerte                    │
│   - ax_k, axn_k (Leibrenten)                           │
│   - nGrAx (Todesfallbarwert)                           │
│   - nGrEx (Erlebensfallbarwert)                        │
└────────────────┬────────────────────────────────────────┘
                 │
                 v
┌─────────────────────────────────────────────────────────┐
│  EXCEL-FORMELN: Beitragsberechnung                     │
│   K5: Bxt (Bruttobeitrag Tod)                          │
│   K9: Pxt (Nettoprämiensatz Tod)                       │
└────────────────┬────────────────────────────────────────┘
                 │
                 v
┌─────────────────────────────────────────────────────────┐
│  EXCEL-FORMELN: Verlaufswerte (Zeilen 16+)             │
│   - Barwerte pro Vertragsjahr                          │
│   - Deckungsrückstellungen                             │
│   - Rückkaufswerte                                     │
└─────────────────────────────────────────────────────────┘
\end{verbatim}
\end{minipage}
\caption{Datenfluss der Tarifberechnung}
\end{figure}

\subsection{Kritische Berechnungsbestandteile}

\subsubsection{Cache-Optimierung}

\begin{itemize}
    \item $D_x$, $C_x$, $N_x$, $M_x$, $R_x$ werden gecacht
    \item Vermeidet Mehrfachberechnung bei Array-Formeln
    \item Cache-Invalidierung: Nur bei Excel-Neustart
    \item Wichtig für Performance bei großen Verlaufstabellen
\end{itemize}

\subsubsection{Array-Formeln}

\begin{itemize}
    \item K5, K9: Berechnungsintensive Hauptformeln
    \item Spalten B--E (Zeilen 16+): Array-Formeln für Verlaufswerte
    \item \textbf{Vorteil:} Kompakte Darstellung, automatische Aktualisierung
    \item \textbf{Nachteil:} Langsam bei vielen Zeilen, schwer zu debuggen
\end{itemize}

\subsubsection{Rundungsgenauigkeit}

\begin{itemize}
    \item Alle VBA-Berechnungen: 16 Dezimalstellen
    \item Excel-Anzeige: Oft weniger Stellen (nur Formatierung)
    \item \textbf{Kritisch:} Kleine Abweichungen können sich akkumulieren
    \item \textbf{Banker's Rounding:} Excel verwendet kaufmännische Rundung
\end{itemize}

%===============================================================================
\section{Anforderungen für Python-Migration}
%===============================================================================

\subsection{Funktionale Anforderungen}

\subsubsection{Pflichtanforderungen}

\begin{enumerate}
    \item Berechnung identischer Ergebnisse wie Excel-Version (Toleranz $< 10^{-6}$)
    \item Unterstützung DAV1994\_T und DAV2008\_T Tafeln (männlich/weiblich)
    \item Alle Barwert-Funktionen:
    \begin{itemize}
        \item Leibrenten: $\ddot{a}_x^{(k)}$, $\ddot{a}_{x:\overline{n}\,}^{(k)}$, ${}_n\ddot{a}_x^{(k)}$
        \item Versicherungen: ${}_nA_x$, ${}_nE_x$
        \item Abzugsglied: $\beta(k, i)$
    \end{itemize}
    \item Beitragsberechnung: $B_{x:t}$, $P_{x:t}$
    \item Verlaufswerte für beliebige Anzahl Vertragsjahre
    \item Validierung zulässiger Zahlungsweisen: zw $\in \{1, 2, 4, 12\}$
\end{enumerate}

\subsubsection{Optionale Anforderungen}

\begin{enumerate}
    \item Caching-Mechanismus für Performance-Optimierung
    \item Validierung aller Eingabeparameter (Wertebereich, Konsistenz)
    \item Export nach Excel/PDF
    \item Grafische Darstellung der Verlaufswerte
    \item Batch-Berechnungen für Parametervariationen
\end{enumerate}

\subsection{Technische Anforderungen}

\subsubsection{Python-Packages}

\begin{table}[h]
\centering
\begin{tabular}{ll}
\toprule
\textbf{Package} & \textbf{Verwendungszweck} \\
\midrule
numpy & Array-Operationen, Vektorisierung, mathematische Funktionen \\
pandas & Datenmanagement, Tafel-Handling, DataFrame-Export \\
openpyxl & Excel-Import/Export (optional) \\
matplotlib & Visualisierung (optional) \\
\bottomrule
\end{tabular}
\caption{Erforderliche Python-Packages}
\end{table}

\subsubsection{Architektur-Prinzipien}

\begin{enumerate}
    \item \textbf{Modularer Aufbau:} Analog zu VBA-Modulen
    \begin{itemize}
        \item Modul \texttt{constants.py}: Konstanten
        \item Modul \texttt{grundwerte.py}: Kommutationswerte
        \item Modul \texttt{barwerte.py}: Leibrenten und Versicherungen
        \item Modul \texttt{tarif.py}: Beitragsberechnung
    \end{itemize}
    
    \item \textbf{Klare Trennung:} Daten -- Logik -- Präsentation
    
    \item \textbf{Testbarkeit:} Unit-Tests für alle Funktionen
    
    \item \textbf{Dokumentation:} Docstrings nach NumPy-Konvention
\end{enumerate}

\subsubsection{Validierungsstrategie}

\begin{enumerate}
    \item Vergleich Python vs. Excel für Standardtestfälle
    \item Toleranz für Rundungsdifferenzen: $\epsilon = 10^{-10}$ (absolut)
    \item Relative Toleranz bei großen Werten: $10^{-6}$
    \item Test-Suite mit verschiedenen Parameterkombinationen
\end{enumerate}

%===============================================================================
\section{Kritische Punkte für Migration}
%===============================================================================

\subsection{Genauigkeitsaspekte}

\subsubsection{Rundung}

\begin{itemize}
    \item \textbf{Excel:} \texttt{WorksheetFunction.Round} verwendet Banker's Rounding
    \item \textbf{Python:} \texttt{numpy.round} verwendet dieselbe Methode (Standard)
    \item \textbf{Kritisch:} Konsistenz prüfen bei Grenzfällen (z.\,B. 0,5 $\to$ nächste gerade Zahl)
\end{itemize}

\subsubsection{Akkumulation von Rundungsfehlern}

\begin{itemize}
    \item \textbf{$l_x$-Werte:} Rekursive Berechnung akkumuliert Rundungsfehler
    \item \textbf{$N_x$, $M_x$, $R_x$:} Summationen vergrößern Abweichungen
    \item \textbf{Lösung:} Hohe Präzision beibehalten (\texttt{np.float64})
    \item \textbf{Rundung:} Erst am Ende der Berechnungskette
\end{itemize}

\subsection{Performance-Aspekte}

\subsubsection{Cache-Mechanismus}

\begin{itemize}
    \item \textbf{Excel:} Dictionary in VBA, Lebensdauer = Sitzung
    \item \textbf{Python:} 
    \begin{itemize}
        \item Decorator \texttt{@lru\_cache} aus \texttt{functools}
        \item Oder dict-basierter Cache in Klasse
    \end{itemize}
    \item \textbf{Wichtig:} Cache-Keys müssen alle Parameter beinhalten
    \item \textbf{Hashable:} Nur immutable Typen als Keys verwenden
\end{itemize}

\subsubsection{Vektorisierung}

\begin{itemize}
    \item \textbf{Vermeidung von Schleifen:} Nutze NumPy-Array-Operationen
    \item \textbf{Vectorize:} \texttt{numpy.vectorize} oder direkte Array-Operationen
    \item \textbf{Profiling:} Mit \texttt{timeit} oder \texttt{cProfile}
    \item \textbf{Beispiel:} Berechnung aller $l_x$ auf einmal statt Schleife
\end{itemize}

\subsection{Usability-Aspekte}

\subsubsection{Eingabe}

\begin{itemize}
    \item \textbf{Excel:} Direkte Eingabe in Zellen
    \item \textbf{Python:} 
    \begin{itemize}
        \item Parameterübergabe an Funktionen
        \item Config-File (JSON, YAML)
        \item Kommandozeilen-Interface (CLI)
        \item GUI (z.\,B. mit Streamlit)
    \end{itemize}
\end{itemize}

\subsubsection{Ausgabe}

\begin{itemize}
    \item \textbf{Excel:} Tabelle mit Verlaufswerten
    \item \textbf{Python:}
    \begin{itemize}
        \item \texttt{pandas.DataFrame} für Verlaufswerte
        \item Excel-Export mit \texttt{openpyxl}
        \item Grafiken mit \texttt{matplotlib}
        \item PDF-Report mit \texttt{reportlab}
    \end{itemize}
\end{itemize}

%===============================================================================
\section{Beispielrechnung (Validierung)}
%===============================================================================

\subsection{Eingabewerte}

\begin{table}[h]
\centering
\begin{tabular}{lll}
\toprule
\textbf{Parameter} & \textbf{Symbol} & \textbf{Wert} \\
\midrule
Eintrittsalter & $x$ & 40 Jahre \\
Geschlecht & Sex & M (männlich) \\
Versicherungsdauer & $n$ & 30 Jahre \\
Beitragszahlungsdauer & $t$ & 20 Jahre \\
Versicherungssumme & VS & 100.000 EUR \\
Zahlungsweise & zw & 12 (monatlich) \\
Rechnungszins & $i$ & 1,75\% \\
Sterbetafel & Tafel & DAV1994\_T \\
\bottomrule
\end{tabular}
\caption{Beispielrechnung -- Eingabeparameter}
\end{table}

\subsection{Erwartete Ergebnisse (aus Excel)}

\subsubsection{Beitragsberechnung}

\begin{table}[h]
\centering
\begin{tabular}{lrl}
\toprule
\textbf{Größe} & \textbf{Wert} & \textbf{Einheit} \\
\midrule
$B_{x:t}$ (K5) & 0,04226001 & je 1 EUR VS \\
BJB (K6) & 4.226,00 & EUR \\
BZB (K7) & 371,88 & EUR (monatlich) \\
$P_{x:t}$ (K9) & 0,04001217 & je 1 EUR VS \\
\bottomrule
\end{tabular}
\caption{Beispielrechnung -- Beitragsergebnisse}
\end{table}

\subsubsection{Verlaufswerte für $k=0$}

\begin{table}[h]
\centering
\begin{tabular}{lrl}
\toprule
\textbf{Größe} & \textbf{Wert} & \textbf{Einheit} \\
\midrule
${}_{30}A_{40}$ & 0,6315923 & je 1 EUR VS \\
$\ddot{a}_{40:30}^{(1)}$ & 21,4202775 & -- \\
$\ddot{a}_{40:20}^{(1)}$ & 16,3130941 & -- \\
${}_0V_{40}^{\text{bpfl}}$ & $-0,0211300$ & je 1 EUR VS \\
\bottomrule
\end{tabular}
\caption{Beispielrechnung -- Verlaufswerte zu Vertragsbeginn}
\end{table}

\subsection{Validierungsstrategie}

\begin{enumerate}
    \item Berechne mit Python-Implementation
    \item Vergleiche mit Excel-Werten
    \item Akzeptiere Abweichung:
    \begin{itemize}
        \item Absolute Toleranz: $|\text{Python} - \text{Excel}| < 10^{-6}$
        \item Relative Toleranz: $\frac{|\text{Python} - \text{Excel}|}{|\text{Excel}|} < 10^{-6}$
    \end{itemize}
    \item Bei Abweichung: Analyse der Ursache (Rundung, Formel, Cache)
\end{enumerate}

%===============================================================================
\section{Zusammenfassung}
%===============================================================================

\subsection{Kernpunkte}

\begin{itemize}
    \item Der Tarifrechner besteht aus 2 Tabellenblättern und 3 VBA-Modulen
    \item Berechnung basiert auf DAV-Sterbetafeln und Kommutationswerten
    \item Array-Formeln in Excel für Beitrags- und Verlaufsberechnungen
    \item Cache-Mechanismus in VBA für Performance-Optimierung
    \item 16 Dezimalstellen Rundungsgenauigkeit in allen Zwischenschritten
\end{itemize}

\subsection{Nächste Schritte für Python-Migration}

\begin{enumerate}
    \item Import der Sterbetafeln als \texttt{pandas.DataFrame}
    \item Implementation der Grundwerte-Funktionen (Modul \texttt{grundwerte.py})
    \item Implementation der Barwert-Funktionen (Modul \texttt{barwerte.py})
    \item Beitragsberechnung (Modul \texttt{tarif.py})
    \item Verlaufsberechnung (Funktion für alle Vertragsjahre)
    \item Unit-Tests mit Excel-Vergleichswerten
    \item Performance-Optimierung (Caching, Vektorisierung)
    \item Dokumentation und Benutzeroberfläche
\end{enumerate}

\subsection{Kontaktinformationen}

Für Rückfragen zur aktuariellen Logik oder Implementierungsdetails:

\begin{itemize}
    \item Dokumentation erstellt: \today
    \item Basis: Tarifrechner\_KLV.xlsm
    \item Python-Version: 3.11+
\end{itemize}

\end{document}
